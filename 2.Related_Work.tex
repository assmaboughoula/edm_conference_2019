\section{Related Work}
%OLD TEXT, NEEDS UPDATING. MORE CITATIONS
One challenge we encountered in this work was the dearth of previous works on methods for automatic concept extraction that might provide us with a previously recognized baseline. While there are plenty of papers in the Educational Data Mining domain which explore modeling relationships between concepts and building "concept dependency graphs", there are few that present methods specific for concept extraction and indexing; most papers we encountered view concept extraction as a preliminary step to the more interesting problem of inferring dependency relationships between concepts, and thus is treated as a less important task with many papers assuming the existence of an expert curated concept set without providing information on how these sets were created. In this section we will look at some of the methods presented in previous concept modeling papers for concept extraction.

Some previous methods for concept extraction include the analysis of grammatical, temporal, structural, and/or semantic relations. 
The authors in \cite{Mirroshandel:12} propose two different algorithms for the semi-supervised and unsupervised extraction of temporal relations of events from within documents, using clustering and expectation maximization respectively. In terms of structural relations, \cite{Paass:11} identifies methods for extracting information from documents based on their geometric and logical layouts. The authors of these papers noted that both methods require knowledge about the structure of the documents beforehand; consequently, these methods are unsuitable for extracting concepts from unstructured texts, such as textbooks and lecture transcripts.
To capture semantic relations, \cite{Liang:15} proposes a metric called \textit{reference distance} for capturing the relatedness of concepts in a prerequisite-dependency relationship between Wikipedia articles. One direction of future work in our paper is to extend and slightly alter the authors' techniques to mine hypernym-hyponym relationships from Wikipedia, which would in turn assist in determining the hierarchical structure of course concepts.
In \cite{AlSaad}, the authors proposed a measure for capturing the dependency relations within and between lectures for a course on Coursera. In their work, the concept space for the Coursera course was built in a semi-automated fashion, but ultimately required extensive and expensive manual grouping and categorization of some of the concepts by experts in the course. One future direction outlined in this past work was to explore unsupervised models for building the concept space.

Another issue we ran into when researching previous works in concept modeling is that often, different papers adopt different definitions and representations of concepts, from phrases and keywords, to topics and word distributions, to Wikipedia article titles and hyperlinks. %provide citations
% MAY BE TOO PRETENTIOUS...
In this work we will attempt to unify concept definition by presenting a novel framework for understanding a concept space and how concepts are represented in it.