\section{Introduction}
As independent learning continues to climb in popularity, it is increasingly important to provide learners with tools to help learners with decision making and independent curriculum planning. One advantage of online education is that flexibly designed learning modules can be adapted to the individual needs. For example, a learner can combine modules from multiple MOOCs in a flexible way. 

However, it can be challenging for learners to select the best/right modules for them to learn for multiple reasons: (1) Unliked a regular course that has a pre-defined structure, such flexibly combined modules may not have pre-defined pre-requisite guidelines and the expected major take-aways, making it a challenge for learners to assess whether they have sufficient background for taking a target module or whether the module offers the expected learning goals that they are looking for, before actually taking the module. The problem is worsen by the fact that learners are of variable background, often not aligned well with the pre-requisite of a target module. 
(2) Their variable background also makes it hard to pre-assess what major concepts they can expect to learn from a particular target module. 

To address these challenges, we propose to automatically build concept profiles for online learning modules, which can help learners quickly decide whether a module is suitble for their needs. A concept profile consists of two kinds of concepts: 1) Elemental Concepts which are basic concepts used in a learning module, but may not be explained in the module; elemental concepts are useful for helping a learner assess whether the learner has the needed background to learn the module. 2) Composite Concepts which are advanced concepts representing the major topics explained in the module; composite concepts can serve as a useful preview of the concepts that a learner can expect to learn from the module.

We propose and study methods for discovering and identifying Elemental and Composite concepts from instructional text data such as textbooks and MOOC video lecture transcripts to build a Concept Profile for a learning module. Evaluation results on several Coursera MOOCs show promising results of the proposed methods, demonstrating the feasibility of automatic construction of useful concept profiles for online learning modules.
